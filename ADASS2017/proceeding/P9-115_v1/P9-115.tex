% This is the ADASS_template.tex LaTeX file, 26th August 2016.
% It is based on the ASP general author template file, but modified to reflect the specific
% requirements of the ADASS proceedings.
% Copyright 2014, Astronomical Society of the Pacific Conference Series
% Revision:  14 August 2014

% To compile, at the command line positioned at this folder, type:
% latex ADASS_template
% latex ADASS_template
% dvipdfm ADASS_template
% This will create a file called aspauthor.pdf.}

\documentclass[11pt,twoside]{article}

% Do NOT use ANY packages other than asp2014. 
\usepackage{asp2014}

\aspSuppressVolSlug
\resetcounters

% References must all use BibTeX entries in a .bibfile.
% References must be cited in the text using \citep{} or \citep{}.
% Do not use \cite{}.
% See ManuscriptInstructions.pdf for more details
\bibliographystyle{asp2014}

% The ``markboth'' line sets up the running heads for the paper.
% 1 author: "Surname"
% 2 authors: "Surname1 and Surname2"
% 3 authors: "Surname1, Surname2, and Surname3"
% >3 authors: "Surname1 et al."
% Replace ``Short Title'' with the actual paper title, shortened if necessary.
% Use mixed case type for the shortened title
% Ensure shortened title does not cause an overfull hbox LaTeX error
% See ASPmanual2010.pdf 2.1.4  and ManuscriptInstructions.pdf for more details
\markboth{Torres-Robledo, Brice\~no, Quint and Sanmartim}{The Goodman HTS Data Reduction Pipeline}

\begin{document}

\title{Spectroscopic Data Reduction Pipeline for the Goodman High Throughput Spectrograph}

% Note the position of the comma between the author name and the 
% affiliation number.
% Author names should be separated by commas.
% The final author should be preceded by "and".
% Affiliations should not be repeated across multiple \affil commands. If several
% authors share an affiliation this should be in a single \affil which can then
% be referenced for several author names.
% See ManuscriptInstructions.pdf and ASPmanual2010.pdf 3.1.4 for more details
\author{Sim\'on Torres-Robledo,$^1$ C\'esar Brice\~no,$^{1, 2}$, Bruno Quint$^1$ and David Sanmartim$^3$
\affil{$^1$SOAR Telescope, La Serena, Regi\'on de Coquimbo, Chile; \email{storres@ctio.noao.edu}}
\affil{$^2$Cerro Tololo Interamerican Observatory, Casilla 603, La Serena, Chile}
\affil{$^3$Gemini Observatory, Casilla 603, La Serena, Chile}}
% \affil{$^3$Institution Name, Institution City, State/Province, Country}}

% This section is for ADS Processing.  There must be one line per author.
\paperauthor{Sim\'on Torres-Robledo}{storres@ctio.noao.edu}{ORCID_Or_Blank}{SOAR Telescope}{Software Development}{La Serena}{Regi\'on de Coquimbo}{Postal Code}{Chile}
\paperauthor{C\'esar Brice\~no}{cbriceno@ctio.noao.edu}{ORCID_Or_Blank}{SOAR Telescope}{Software Development}{La Serena}{Regi\'on de Coquimbo}{Postal Code}{Chile}
\paperauthor{Bruno Quint}{bquint@ctio.noao.edu}{ORCID_Or_Blank}{SOAR Telescope}{Software Development}{La Serena}{Regi\'on de Coquimbo}{Postal Code}{Chile}
\paperauthor{David Sanmartim}{dsanmartim@gemini.edu}{ORCID_Or_Blank}{Gemini Observatory}{Software Development}{La Serena}{Regi\'on de Coquimbo}{Postal Code}{Chile}

\begin{abstract}
The Goodman High Throughput Spectrograph (Goodman Spectrograph) is a highly
versatile instrument in operation at the SOAR Telescope on Cerro Pach\'on, Chile.
It is capable of doing low to mid-resolution spectroscopy in a range from 3200 \AA{}
to 9000 \AA{}. The data reduction pipeline is conceived as an easy-to-run software,
that can process an entire night worth of data by execution of a simple command,
with some arguments, from a terminal window. It is written almost entirely in
Python, following several Python standard conventions.
We aim towards using exclusively standard Python packages, such as Astropy and
Astropy-affiliated packages, while at the same time allowing for fast and
efficient computing.
In its present form the pipeline produces fully reduced, wavelength-calibrated
spectra. Flux calibration will be an add-on option for a later release.
\end{abstract}

\section{Introduction}

The Goodman Spectrograph \citep{2004SPIE.5492..331C} is a highly customizable spectrograph in operation at
the 4-meter SOAR Telescope on Cerro Pach\'on, northern Chile.
It is capable of delivering spectra in a wavelength range from 3200 \AA{} to 9000 \AA{}
with resolving power ($R=\lambda/\Delta\lambda$) ranging from 800 to 14,000
depending on the configuration. The Goodman Spectrograph is the workhorse
instrument of the SOAR Telescope, with about 65\% of the available time on the sky.
Hence the need of a dedicated pipeline.
Moreover, because SOAR is on a path toward increased automation, looking forward
to be an efficient follow up facility in the LSST era, an automated pipeline
becomes a critical requirement.

Although IRAF is considered by many the standard tool
in data reduction, its scope is to provide data reduction and processing
capabilities to a rather wide variety of cases, thus is a bit hard to implement
as an automatic pipeline.  We chose Python because of its
versatility, completeness and because it allows for fast development and
robust products. The existence and maturity of Astropy and its affiliated
packages was also a big motivation to choose Python.
%Our initial plan considered a Python-only project but we had to change that
%because LaCosmic \citep{2001PASP..113.1420V} did not provide satisfactory results and we had to shift
%to using a program called DCR \citep{2004PASP..116..148P} which is written in C,
%it works very well but does not integrate into Python easily, ideally this
%should be converted into a python C++ extension but it is not a high priority
%task.
%LaCosmic implementation is still part of the pipeline and can be selected with a command
%line argument.

%Also we wanted the pipeline to be open source and available for everyone; in fact the full
%project is available on GitHub.

The Goodman Spectroscopic Pipeline will provide clear advantages to users of the
Goodman Spectrograph compared to other tools for data reduction by allowing
them to run full processing of their spectra with one-line command while producing
science-quality data.

%Compared to other tools for data reduction the Goodman Spectroscopic Pipeline
%will provide clear advantages to users of the Goodman Spectrograph, by allowing
%them to run full processing of their spectra with a one-line command, that produces
%wavelength-calibrated, science quality results.
%Ours is not the first or only
%effort in providing a pipeline for reducing Goodman spectra.
There have been other efforts to develop a data reduction pipeline for Goodman
Spectrograph. However, this is the first one developed not as a solution for one
specific project, but rather as a community tool. It has been conceived to be
simple to use and very well documented, so users can clearly understand each
processing step, and even further develop and modify the software.
The project is hosted on GitHub therefore is freely available.

\section{Pipeline Concept \& Structure}

%The pipeline started as two parallel projects, one for image reduction
%and one for spectra reduction. Very soon we realized they could not be isolated
%but should be part of an integrated system. They evolved into a single project with two main
%packages that work together.
In the current stable version (1.0b1) the pipeline is structured as two packages
with two scripts that act as the user interface through a console or terminal.
The motivation for having two packages is to avoid duplicate work, since
the Goodman Spectrograph is actually an \emph{imaging spectrograph} the idea is
that one of the packages contain tools for processing \emph{Imaging} data and
the 2D part of the \emph{Spectroscopic} data, and the other will continue with
the specific steps for spectroscopy only.

%The static structure still needs some refactoring and for sure there will be
%modifications to the dynamic structure.

% Figure \ref{pipestruct} is a representation of the dynamic structure,  
% with  arrows indicating \emph{imports from}.
% 
% 
% \articlefigure{P8989_f1.pdf}{pipestruct}{Dynamic structure of the
% pipeline in its current version 1.0b1. The arrows mean ``imports from''. This does not represent a workflow.}

For the next release (1.0b2) we are developing a single-package structure that will
contain sub-packages including the current two as well as new features, such as
a telemetry package that can be used to monitor the instrument performance by
comparing wavelength calibration parameters.
The split-process idea remains but the code is more condensed and better structured.
%Another very important feature that we are developing is integrated code testing.

\section{Pipeline Use}
The Goodman Pipeline has been developed to be easy to use. There are two scripts
or commands that you have to run in order to get fully processed data:
\verb=redccd= and \verb=redspec=.
\verb=redccd= is for the initial image processing, such as bias
subtraction and flat correction and \verb=redspec= is for the spectroscopic
part; identification, trace, extraction and wavelength calibration.
Both scripts can be called with no arguments and should work well for most general use cases
and well behaved data, but there is also the possibility of customizing the behavior
up to a certain level. Cosmic Ray rejection is done with the DCR program,
written in C \citep{2004PASP..116..148P}. Though our initial plan was to
implement a Python-only project, we had to adopt LaCosmic
\citep{2001PASP..113.1420V} did not provide satisfactory results for our
spectroscopic data. However, LaCosmic is also implemented in the pipeline and
can be selected with a command line argument. Although the execution sequence
was gracefully integrated into the pipeline's workflow there is a rigidity
regarding parameter variation, because it uses a parameter file with a specific
name and at a specific location.

% For parsing command line arguments we use the standard python library \verb=argparse= which
% also allows easy integration with other ways of calling the script, from a GUI 
% for instance or even from a Jupyter Notebook. Also, it provides nicely formatted 
% help text upon argument failure or by using \verb=-h= or \verb=--help=

\subsection{redspec}

% The spectroscopic reduction code was probably the most complex to develop, given
% that Astropy's libraries were not ready at the moment. It starts by classifying 
% the spectroscopic data, the output from \verb=redccd=, using a set of keywords 
% from the header, an important point to highlight here is that this pipeline does not
% rely on file names to classify the spectra, it only uses prefixes in order to filter
% data and for the user to easily keep track of the processing done, otherwise, it
% trusts that the header information is correct. As a result of the classification 
% process it creates an object whose attributes contain information regarding the location
% of the data, time reference points, such as, the sunset and sunrise time. It also
% contain lists of \verb=pandas.DataFrame= instances that group all the images that
% ``belong together'' according to the classification scheme. It will consider three
% possible scenarios: A comparison lamp that is not associated with any science target.
% A science target that does not have associated any comparison lamp.
% Groups of science targets and comparison lamps that are related to each other.
% 
% The target identification works by detecting the n-most intense peaks in the spatial
% profile obtained by collapsing the spectrum image along the dispersion direction using
% a median average. If there are any background features they are removed assuming the background 
% should be flat. An equivalent implementation should be done for extended sources.
% 
% The tracing of the target is done by keeping track of the path of the spectrum's highest
% point. This is one of the parts that need more development in order to work for fainter
% sources or sources near very bright neighbors.
% 
% The current stable version has a simple extraction method only, meaning, it simply
% sums the counts along the spatial direction within a rectangular region that
% fully contains the spectrum.
% A background subtraction is done by selecting a clear zone outside this rectangular
% region and with the same number of pixels, if it is possible to get the background
% level at both sides of the spectrum, an average of both background regions is 
% calculated and if only one region is possible it is averaged and multiplied by
% the width in pixels of the rectangular region that contains the spectrum.

Perhaps one of the most interesting parts of the code is the interactive and
automatic wavelength calibration. It relies on a library of previously calibrated
lamps, which makes things extremely easy compared to other methods, like
\verb=identify= in IRAF.
The interactive mode was developed as a way around the complications that the automatic
wavelength calibration meant at that early stage of development. It allowed us to 
understand how the data behaved in the wavelength regime. Also, and more importantly,
it gave us useful insight that allowed us to later implement the automatic wavelength solution.
In order to obtain a wavelength solution in interactive mode, the user has to click-identify 
the lines in two plots, one with a reference lamp plus reference line values and
another with the new lamp and the detected lines marked. The clicks are stored in
two lists that later are used to calculate the fit of a low order Chebyshev polynomial.
In Python this results in a callable object to wich you can assign a pixel axis as argument
and will return an axis in Angstrom. The use of the reference lamp is only for
\emph{visual reference}.

The automated wavelength calibration instead detects emission lines in the comparison lamps
and then splits the spectrum in an even number of parts according to the number of lines detected.
The reference lamp is also split. Equivalent portions are then cross-correlated,
producing an offset value for the corresponding center value, still in pixels,
of the lines detected in the new comparison lamp. Using the mathematical model of the reference
lamp's wavelength-solution  the equivalent line center in angstrom is obtained 
using the equation $ \lambda_{\AA} = model(pixel + offset)$


As a first order filter for mismatched lines the cross-correlation values, which
are stored separated, will be sigma-clipped using 2-sigmas and one-iteration only.
This eliminates the most obvious mismatched lines. The rejected values also reject
their pixel and angstrom equivalents, producing a cleaner new wavelength solution.


Using the Angstrom values previously found and the detected lines
plus the newly calculated solution, the differences in angstrom are
calculated, and a new sigma-clipping is applied, since the distributions are not
necessarily normal distributions.

Once these values are cleaned of rejected values the final solution
is calculated using a low order Chebyshev polynomial.


% For instance, if an 
% observer took five exposures to a given target, they will be grouped together.
% The same if for instance, at a given pointing the observer took, quartz lamp,
% comparison lamps and a science exposure.

\section{Results}
% \section{Conclusion}
The pipeline has met our expectation of being an efficient and fast
spectroscopic data processing tool. Although its scientific usefulness has not
been validated yet, we believe we are close because the results are
repeatable and consistent. Since there is no other production-ready pipeline
available for the Goodman Spectrograph data, at the moment we can only compare it to
IRAF. Figure \ref{irafcomp} contains the result obtained for the same spectrum using
the pipeline in automatic mode versus IRAF, by two different users. The good
match is evident even when zooming in.
We find that, using the 400 l/mm grating with the $1\arcsec$ slit, which yields
a resolution R $\sim 800$ (FWHM $\sim 6 \AA$), the pipeline produces an automated 
wavelength solution with RMS $\sim 0.3 \AA$. This is roughly the same RMS obtained
with IRAF.

\articlefigure{P9-115_f1.eps}{irafcomp}{Results obtained using
the Goodman Pipeline compared to IRAF. There is an arbitrary vertical offset to
clearly set apart both results. The zoomed portion focuses on $H\alpha$, 
in emission, also allowing to view nearby features.}



\section{Future Work}

At the moment of writing, the pipeline is in its first beta version and as
expected there are some features still missing. They should appear in future
releases. Examples are flux calibration, and a complete implementation of fractional pixel
extraction and optimal extraction \citep{1989PASP..101.1032M} and \citep{1986PASP...98..609H}.
We also want to change the Matplotlib UI by Qt UIs.
An important development looking forward will be the implementation of an online
version of the pipeline, that can output reduced, wavelength calibrated 1-D
spectra just seconds after the shutter has closed and the raw data saved to
disk.

Other planned changes are: to modify the overall architecture of the pipeline,
since it started as two separated projects and evolved into a single one, there
is a development version that integrates everything as a single package. The 
ultimate goal in this regard is to make it comply with the Astropy Affiliated
packages standard although we need to evaluate how quickly we want that.

Talking about Astropy, there should be a replacement of tasks that are becoming
available in Astropy and Astropy Affiliated Packages as they mature, wavelength
solution storage for instance.

On the coding side itself we must implement test code and in fact we might switch
to a \emph{test driven development} philosophy.

\acknowledgements  This research made use of Astropy, a community-developed core Python package for Astronomy (Astropy Collaboration, 2013). IRAF is distributed by the National Optical Astronomy Observatories, which are operated by the Association of Universities for Research in Astronomy, Inc., under cooperative agreement with the National Science Foundation.


\bibliography{biblio}  % For BibTex

\end{document}
